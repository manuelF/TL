\documentclass[a4paper]{article}

\usepackage[spanish]{babel}
\usepackage[utf8x]{inputenc}
\usepackage{amsmath}
\usepackage{graphicx}
\usepackage[colorinlistoftodos]{todonotes}

\title{Trabajo Práctico \\ Teoría de Lenguajes}


\author{Manuel Ferreria, Luciano Gandini, Ignacio Gleria}

\begin{document}
\maketitle



\section{Introducción}

Your introduction goes here! Some examples of commonly used commands and features are listed below, to help you get started.

If you have a question, please use the support box in the bottom right of the screen to get in touch. 

\section{Símbolos}


\subsection{Símbolos no terminales}

\begin{center}
$V_n$ = \{S R op G F H I N M P funcion\}.
\end{center}

\subsection{Símbolos terminales}

\begin{center}
$V_t$ = \{, \{ \} ( ) num expand reduce post loop fill tune play sin lin sil noi mix con add sub mul div\}.
\end{center}

\subsection{Símbolo distinguido}

\begin{center}
$S$ = S 
\end{center}

\section{Producciones}

Del símbolo inicial a una lista de lista generadores con aplicaciones o generadores con aplicaciones unidos por operandos.
\begin{center}
S $\Rightarrow$ \{S\}N $|$ GNR \\
R $\Rightarrow$ $\lambda$ $|$ op B 
\end{center}

Las operaciones pueden ser invocadas con los símbolos o con los nombres completos.
\begin{center}
op $\Rightarrow$ mix $|$ con $|$ add $|$ sub $|$ mul $|$ div \\ 
\end{center}

Los generadores pueden ser o numeros u objetos devueltos por funciones
\begin{center}
G $\Rightarrow$ numero $|$ funcion \\ 
\end{center}





\subsection{Comments}

Comments can be added to the margins of the document using the \todo{Here's a comment in the margin!} todo command, as shown in the example on the right. You can also add inline comments too:

\todo[inline, color=green!40]{This is an inline comment.}

\subsection{Tables and Figures}

Use the table and tabular commands for basic tables --- see Table~\ref{tab:widgets}, for example. You can upload a figure (JPEG, PNG or PDF) using the files menu. To include it in your document, use the includegraphics command as in the code for Figure~\ref{fig:frog} below.

% Commands to include a figure:
\begin{figure}
\centering
\includegraphics[width=0.5\textwidth]{frog.jpg}
\caption{\label{fig:frog}This is a figure caption.}
\end{figure}

\begin{table}
\centering
\begin{tabular}{l|r}
Item & Quantity \\\hline
Widgets & 42 \\
Gadgets & 13
\end{tabular}
\caption{\label{tab:widgets}An example table.}
\end{table}

\subsection{Mathematics}

\LaTeX{} is great at typesetting mathematics. Let $X_1, X_2, \ldots, X_n$ be a sequence of independent and identically distributed random variables with $\text{E}[X_i] = \mu$ and $\text{Var}[X_i] = \sigma^2 < \infty$, and let
$$S_n = \frac{X_1 + X_2 + \cdots + X_n}{n}
      = \frac{1}{n}\sum_{i}^{n} X_i$$
denote their mean. Then as $n$ approaches infinity, the random variables $\sqrt{n}(S_n - \mu)$ converge in distribution to a normal $\mathcal{N}(0, \sigma^2)$.

\subsection{Lists}

You can make lists with automatic numbering \dots

\begin{enumerate}
\item Like this,
\item and like this.
\end{enumerate}
\dots or bullet points \dots
\begin{itemize}
\item Like this,
\item and like this.
\end{itemize}

We hope you find write\LaTeX\ useful, and please let us know if you have any feedback using the help menu above.

\end{document}
