\documentclass[a4paper]{article}

\usepackage[spanish]{babel}
\usepackage[utf8x]{inputenc}
\usepackage{amsmath}
\usepackage{graphicx}
\usepackage[colorinlistoftodos]{todonotes}

\title{Trabajo Práctico \\ Teoría de Lenguajes}


\author{Manuel Ferreria, Luciano Gandini, Ignacio Gleria}

\begin{document}
\maketitle



\section{Introducción}

Your introduction goes here! Some examples of commonly used commands and features are listed below, to help you get started.

If you have a question, please use the support box in the bottom right of the screen to get in touch. 

\section{Símbolos}

\begin{itemize}
\item \textbf{Símbolos no terminales}

\begin{center}
$V_n$ = \{S R op G F H I N M P funcion\}.
\end{center}

\item \textbf{Símbolos terminales}

\begin{center}
$V_t$ = \{, \{ \} ( ) num expand reduce post loop fill tune play sin lin sil noi mix con add sub mul div\}.
\end{center}

\item \textbf{Símbolo distinguido}

\begin{center}
$S$ = S 
\end{center}
\end{itemize}

\section{Producciones}
\begin{itemize}
\item Del símbolo inicial a una lista de lista generadores con aplicaciones o generadores con aplicaciones unidos por operandos.:
\begin{center}
S $\Rightarrow$ \{S\}N $|$ GNR \\
R $\Rightarrow$ $\lambda$ $|$ op B 
\end{center}

\item Las operaciones pueden ser invocadas con los símbolos o con los nombres completos:
\begin{center}
op $\Rightarrow$ mix $|$ con $|$ add $|$ sub $|$ mul $|$ div \\ 
\end{center}

\item Los generadores pueden ser o numeros u objetos devueltos por funciones:
\begin{center}
G $\Rightarrow$ numero $|$ funcion \\ 
\end{center}

\item Las funciones según definidas en la página 2 del enunciado devuelven objetos:
\begin{center}
funcion $\Rightarrow$  FH \\
F $\Rightarrow$ sin $|$ lin $|$ sil $|$ noi \\
H $\Rightarrow$ $\lambda$ $|$ ( num I \\
I $\Rightarrow$ ,num) $|$ ) \\
\end{center}

\item Los métodos de alteración pueden tener parámetros o no, y pueden ser encadenados:
\end{itemize}
\begin{center}
N $\Rightarrow$ lambda $|$ .MN \\
M $\Rightarrow$ expand $|$ reduce $|$ post $|$ play P $|$ loop( num ) $|$ fill( num ) $|$ tune( num ) \\
P $\Rightarrow$ $\lambda$ $|$ ( num ) \\
\end{center}

\section{Lexer}

\subsection{Presunciones }
\begin{itemize}
\item Nos reemplaza los números por el token numeros usando la expresión regular definida en la regla 'num'.
\item Nos reemplaza las funciones por sus tokens correspondientes.
\item Nos elimina todo caracteres que sigue el $//$ hasta el \textbackslash$n$, incluyendo las $//$
\end{itemize}

\subsection{Configuración del Lexer }
\begin{itemize}

\item Los números los definimos con expresiones regulares:
\begin{center}
num $\equiv$ (+ $|$ - $|$ ) [[1..9][0..9]* $|$ 0 ][.[0..9]*[1..9] $|$ .0 $|$ ]
\end{center}

\item Las funciones definidas con expresiones regulares:
\begin{center}
sin $\equiv$ sin \\ 						
lin $\equiv$ [lin $|$ linear] \\
sil $\equiv$ [sil $|$ silence] \\
noi $\equiv$ [noi $|$ noise] \\
\end{center}

\item Los operandos definidos con expresiones regulares:
\begin{center}
mix $\equiv$ \& $|$ mix \\
con $\equiv$ ; $|$ con \\
add $\equiv$ + $|$ add \\
sub $\equiv$ - $|$ sub \\
mul $\equiv$ * $|$ mul \\
div $\equiv$ / $|$ div
\end{center}

\item Reducción del whitespace y eliminación de los comentarios:

\begin{center}
E $\equiv$ [//\textbackslash$w$*\textbackslash$n$ $|$ \textbackslash$s$ $|$ \textbackslash$n$ ]*  
\end{center}

\end{itemize}

\section{Decisiones tomadas}
\begin{itemize}
\item Todas las funciones  pueden tomar 0, 1 o 2 parámetros, tomando valores por defecto en caso de que sean menos de 2 parámetros. En el caso de tener 0 parametros, no se escriben los paréntesis '()'.

\item El whitespace de los programas es eliminado y no considerado, con excepción del caso de las //, que eliminan lo siguiente de esa línea.

\item La gramática definida es LL(1) dado que para dos producciones con misma parte izquierda, son disjuntos los simbolos directrices generados.
\end{itemize}

\section{Conclusión}
A traves de este trabajo, pudimos apreciar las diferencias entre lexers y parsers. Aunque tomamos un subconjunto de las reglas necesarias para escribir una gramática que sea entendible por un compilador,
hace falta un poco más de trabajo para que sea entendible por un compilador como \texttt{ANTLR} o \texttt{yacc}.  \\
Quizás haber hecho que sea LL(1) lo hace un poco más ilegible frente a otras decisiones que podríamos haber tomado, pero nos pareció didácticamente conveniente usarlo.

\end{document}
