\documentclass[a4paper]{article}

\usepackage[spanish]{babel}
\usepackage[utf8x]{inputenc}
\usepackage{amsmath}
\usepackage{graphicx}
\usepackage[colorinlistoftodos]{todonotes}
\usepackage{hyperref}
\usepackage{xcolor}
\usepackage{listings}

\lstset{% This applies to ALL lstlisting
%        backgroundcolor=\color{yellow!10},%
            numbers=left, numberstyle=\tiny, stepnumber=1, numbersep=2pt,%
                }%

                % Applies only when you use it
                \lstdefinestyle{MyLang}{
                basicstyle=\small\ttfamily\color{magenta},%
                breaklines=true,%                                      allow line breaks
                moredelim=[s][\color{green!50!black}\ttfamily]{'}{'},% single quotes in green
                moredelim=*[s][\color{black}\ttfamily]{options}{\}},%  options in black (until trailing })
                commentstyle={\color{gray}\itshape},%                  gray italics for comments
                morecomment=[l]{//},%                                  define // comment
                emph={%
                        STRING%                                            literal strings listed here
                        },emphstyle={\color{blue}\ttfamily},%              and formatted in blue
                        alsoletter={:,|,;},%
                        morekeywords={:,|,;},%                                 define the special characters
                  keywordstyle={\color{black}},%                         and format them in black
               }

\title{Trabajo Pr\'actico 2 \\ Teoría de Lenguajes}


\author{Manuel Ferreria, Luciano Gandini, Ignacio Gleria}

\begin{document}
\maketitle


\section{El problema}
El trabajo practico n°2 consiste en realizar una gram\'atica para un lenguaje
existente (Supercollider \url{supercollider.sourceforge.net }) y realizar un parser que pueda leerlo, entenderlo
y reproducirlo. El problema esta compuesto principalmente de dos secciones:
\begin{enumerate}
    \item Construir una gram\'atica que sea interpretable usando alguno de los
        compiladores de compiladores (ANTLR, YACC, Bison, etc).
    \item Definir una semantica unida a la gram\'atica, que permita dar significado
        a las producciones, para que se simplifique el trabajo de analisis de las
        cadenas y se llegue a la reproducci\'on adecuada.
\end{enumerate}

\section{Correcciones al TP 1}
Del trabajo pr\'actico anterior al actual se cambi\'o casi en su totalidad la 
parte de la gram\'atica correspondiente al parser y a su vez se cambi\'o la 
expresi\'on que denotaba a los n\'umeros de punto flotante. Se establecieron 
las precedencias requeridas por el enunciado en lo relacionado a operaciones 
aplciadas a los buffers, operaciones aritm\'eticas y generaci\'on de buffers.

\section{Implementaci\'on}
Como habiamos definido para el TP1 una soluci\'on ELL(1), decidimos proceder con
ella para la realizacion de este TP2. Por lo tanto, la mejor herramienta con la
que contamos era ANTLR. 

La estructura que definimos cuenta con tres elementos principales:
\begin{itemize}
    \item \textbf{collider.g}: La gramatica ELL(1) parseable por ANTLR,
        donde se encuentran definidas las producciones y la sem\'antica
        que se asocia.
    \item \textbf{Buffer.java}: Las operaciones que se realizan en la 
        gram\'atica invocan a los m\'etodos estaticos definidos en esta
        clase. Ac\'a se encuentran las operaciones que operan sobre el 
        buffer, los generadores y la reproducci\'on de los sonidos.
    \item \textbf{Main.java/Prompt.java}: ANTLR genera, dada una gram\'atica, un
        par de clases que representan el Lexer y el Parser. Para
        poder reconocer una cadena, hay que instanciar estas clases
        con los strings de input. Ac\'a se las instancia para que se 
        reproduzcan durante el parseo.
\end{itemize}

Para implementar la solucion, lo primero que hicimos fue definir toda la
gr\'amatica, sin la parte de sintaxis. Para esto, nos basamos en el TP1 y
acomodamos nuestras producciones al formato de Lexer y Parser que consume ANTLR.
Para comprobar que se generaban los arboles que nosotros queriamos, utilizamos
herramientas como ANTLRWorks y los plugins de IDEA y Eclipse para intepretar 
las cadenas y ver las derivaciones. 

Luego, agregamos la instanciaci\'on de de las clases generadas por ANTLR
a una clase Main, donde nos permitio empezar a probar cadenas en un 
entorno Java, donde se iba a parsear definitivamente.

Para agregar la sem\'antica, primero tuvimos que entender como se interfaceaba
ANTLR con un lenguaje de programaci\'on (en nuestro caso Java). Luego, descubrimos
como se le agregaban valores de retorno a las producciones, para poder empezar
a pasarselos y recibirlos de m\'etodos de la clase Buffer, que hacian las cuentas
descriptas en el enunciado (\texttt{sin, expand, resample, etc}). Con esto
pudimos describir los atributos sintetizados apropiadamente

Finalmente, el \'ultimo detalle que necesitabamos era como pasar atributos a
producciones, es decir, atributos heredados. Con esto, logramos pasar
los buffers generados a los m\'etodos que los alteraban y reproducian.

\section{Programas ejemplos}
\begin{itemize}
    \item \textbf{Tambores}:\\
        \hbox{\texttt{ \{\{ sin(8); sin(4); sin(2); sin(1) \} * lin( 1.0, 0.0) \}.expand(12).play }}
    \item \textbf{Snare drum}:\\
        \hbox{\texttt{ \{ \{ sin( 16, 0.9)+noise(0.1) \}*lin( 1.0, 0.1) \}.expand(12).play  }}
\end{itemize}
\section{Entradas inv\'alidas}
\begin{itemize}
    \item \textbf{no cierra llaves}:\\
        \hbox{\texttt{ \{\{ sin(8)}} 
    \item \textbf{métodos no aplicados a un buffer}:\\
        \hbox{\texttt{.expand(12).play sin(1)  }}
    \item \textbf{dos generadores seguidos, sin concatenar o que no sean parte de operaciones}:\\
        \hbox{\texttt{ sin( 16, 0.9) noise(0.1)  }}
\end{itemize}

\section{Manual de uso}
Para correr el programa, se debera compilar la gramatica con ANTLR, el Main con \texttt{javac} y luego
se lo debera ejecutar como un \texttt{jar}  de Java normal, incluyendo al jar de ANTLR. Es decir (extrayendo del
\texttt{./execute\_linux.sh}): 
\begin{verbatim}
#! /bin/bash
echo "===Compilando la gramatica==="
java -jar antlr-3.3-complete.jar collider.g;
echo "===Compilando el main==="
javac -cp .:antlr-3.3-complete.jar Main.java;
echo "===Corriendo el main==="
java -cp .:antlr-3.3-complete.jar Main;
echo "===Borrando los .class==="
rm -f *.class;
\end{verbatim}
A la linea de correr el main, tambien se le puede pasar parametros (por 
defecto corre uno de los ejemplos que estan hardcodeados en el main). 
Es decir, se lo puede invocar asi:
\begin{verbatim}

echo "===Corriendo el main con parametros==="
java -cp .:antlr-3.3-complete.jar Main "{sin(44)}.loop(30).play";
\end{verbatim}

Para probar facilmente, hicimos un prompt interprete. Basta con ejecutar
\texttt{./execute\_linux\_prompt.sh}) y genera una linea de comandos donde se puede probar la gram\'atica.

\section{Manual de instalaci\'on}
Lo unico que se requiere para la instalaci\'on, es tener presente Java (1.6 basta),
y el \texttt{.jar} de antlr 3.3, que viene incluido en la entrega.

De no contarse con Java instalado, se lo podra instalar haciendo (en Debian/Ubuntu)
\hbox{\texttt{sudo apt-get install openjdk-6-jdk openjdk-6-jre }}

\section{Decisiones tomadas}
La principal decision que tomamos fue la de no usar objetos adicionales para
recorrer el arbol generado por ANTLR e intepretarlo, y en vez decidimos hacer
todo lo que pedia el enunciado en la semantica. Esto, nos ocasio\'o una gram\'atica
un poco m\'as complicada de lo que podria haber sido si no hubieramos hecho esto,
pero creemos que se entiende lo suficiente y nos deja como ventaja el hecho
de que la gram\'atica quedara autocontenida y no desperdigada en la estructura
de varios archivos java distintos.


Otra decisi\'on que tomamos fue que los numeros pueden ser unicamente de la forma
[0-9]+'.'[0-9]+ o [0-9]. No consideramos  la posiblidad de [0-9]'.' ya que es
normalmente mal vista en varios lenguajes de programaci\'on, porque visibilemente
oculta la existencia de que es un punto flotante y no un entero. Esta decisi\'on en 
particular fue tomada, debido a que el parser no puede diferenciar un n\'umero de 
punto flotante de un n\'umero sin punto, seguido de una aplicaci\'on de un m\'etodo. 
Por lo que se tuvo que delegar el trabajo al parser, haciendo esto mas engorroso 
diferenciar los distintos casos borde de un n\'umero de punto flotante flotante: 
ceros innecesarios adelante y detraz, adem\'as de la aparici\'on del punto sin 
valores en uno de los lados.


\section{Conclusiones}
Finalmente, fue bastante lo que aprendimos con estos dos trabajos practicos.
Creemos que lo m\'as importante fue el hecho de que muchas veces tenemos que
leer archivos que tienen una sintaxis definible de fondo y terminamos haciendo
siempre las mismas soluciones ad-hoc, con todos los bugs que presentan. Con
herramientas como las gr\'amaticas libres de contexto encima, podemos representar
facilmente y de manera muy clara que problema estamos resolviendo.

Otra gran ventaja que descubrimos con esto, es la posiblidad de que la herramienta
(en este caso ANTLR) genere codigo para reconocer cadenas en muchos lenguajes
diferentes (C, Python, Java, Ruby, etc.) Esto nos permite portear al menos 
esta secci\'on de la soluci\'on de una manera eficaz, sencilla y repetible. 
Adem\'as, nos evita potencialmente asumir cosas sobre como procesa el lenguaje
subyacente las cadenas. ANTLR se encarga de manipular las sutilezas y nosotros
del contenido de la cadena leida.


\section{Ap\'endice: C\'odigo}
\subsection{collider.g}
\begin{lstlisting}[style=MyLang]
grammar collider;

options {
    language = Java;
    output = AST;
}

@parser::members {
  public static double getDouble(String text){
    return Double.parseDouble(text);
  }
  private static void print(Object a) {
        System.out.println(a);
  }
}

/* Reglas de Lexer */
/* Nos comemos el whitespace */
WHITESPACE : ( '\t' | ' ' | '\r' | '\n'| '\u000C' )+    { $channel = HIDDEN; } ;
/* Homogenizamos los nombres de las funciones */
SIN: 	'sin' | 'SIN' | 'Sin'
	;

LIN:	'lin' | 'LIN' | 'Lin'
	|'linear'| 'LINEAR' | 'Linear'
	;

SIL:	'sil' | 'SIL' | 'Sil'
	|'silence' | 'SILENCE' | 'Silence'
	;

NOI:	'noi' | 'NOI' | 'Noi'
	|'noise' | 'NOISE' | 'Noise'
	;
/* op:	mix | con | add | sub | mul | div*/
/* Tokenizamos los simbolos de operaciones */
MIX:	'mix' | '&'
	;
CON:	'con' | ';'
	;
ADD:	'add' | '+'
	;
SUB:	'sub' | '-'
	;
MUL:	'mul' | '*'
	;
DIV:	'div' | '/'
	;
/* Definimos los numeros enteros con la regex */
INT	:('0'..'9'('0'..'9')*)
	;
PERIOD:	'.'
	;

PR_START: '('
	;
PR_END:	')'
	;

BR_START: '{'
  ;
BR_END: '}'
  ;
/* Reglas de Parser */ 

/* La gramatica es un buffer con varios metodos aplicados. hubo que factorizar */ 

sGram returns [ArrayList<Double> value]: 
  b=buffer EOF 
    {$value = $b.value;}
  ;

	
buffer returns [ArrayList<Double> value]:	
  b1=mixExpr {$value = $b1.value;} 
    (CON b2=mixExpr {$value = Utils.Concatenate($value,$b2.value);})* 
  ;

mixExpr returns [ArrayList<Double> value]:
  b1=sumaRestaExpr {$value = $b1.value;} 
    (MIX b2=sumaRestaExpr {$value=Buffer.oper("m",$value,$b2.value);})* 
  ;

sumaRestaExpr returns [ArrayList<Double> value]:
  b1=mulDivExpr {$value = $b1.value;} 
    (op=sumaResta b2=mulDivExpr {if ($op.value) $value=Buffer.oper("+",$value,$b2.value); 
                                else $value=Buffer.oper("-",$value,$b2.value);})* 
  ;
  
mulDivExpr returns [ArrayList<Double> value]:
  b1=bufferAtom {$value = $b1.value;} 
    (op=mulDiv b2=bufferAtom {if ($op.value) $value=Buffer.oper("*",$value,$b2.value); 
                             else $value=Buffer.oper("/",$value,$b2.value);})*
  ;
  
bufferAtom returns [ArrayList<Double> value]:
  gen=generador sec1=sec_metodos[$gen.value] {$value = $sec1.value;}
  |
  BR_START b=buffer BR_END sec2=sec_metodos[$b.value] {$value = $sec2.value;}  
  ;
  
sumaResta returns [boolean value]:
  ADD {$value = true;} 
  |
  SUB {$value = false;}
  ;
        
mulDiv returns [boolean value]:
  MUL {$value = true;} 
  |
  DIV {$value = false;} 
  ;
  



generador returns [ArrayList<Double> value]:
  sin {$value = $sin.value;}
  |
	lin {$value = $lin.value;}
	| 
	sil {$value = Buffer.sil();} 
	| 
	noi {$value = $noi.value;}
	|
	a=num {$value = Buffer.buffer(getDouble($a.text));}
	;
	
sin returns [ArrayList<Double> value]: 
  SIN PR_START c=num (PR_END | ',' a=num PR_END) 
        {if($a.text == null){ $value = Buffer.sin(getDouble($c.text),1); } 
          else { $value = Buffer.sin(getDouble($c.text),getDouble($a.text)); }}
  ;
  
  
lin returns [ArrayList<Double> value]:
  LIN PR_START a=num ',' b=num PR_END 
    {$value = Buffer.lin(getDouble($a.text), getDouble($a.text));}
  ;
  
sil returns [ArrayList<Double> value]: 
  SIL {$value = Buffer.sil();}
  ;
  
noi returns [ArrayList<Double> value]:
  NOI PR_START a=num PR_END {$value = Buffer.noi(getDouble($a.text));}
  ;
   
  
/* Instanciamos un metodo que puede recibir parametros */	
sec_metodos[ArrayList<Double> buffInicial] returns [ArrayList<Double> value]:	
	'.' m=metodo[buffInicial] s=sec_metodos[m.value] {$value = $s.value;}
	| {$value = $buffInicial;}
	;

/* Los metodos que reciben parametros, algunos obligatorios */
metodo[ArrayList<Double> buff] returns [ArrayList<Double> value]:	
  'expand' param{$value = Buffer.expand($buff, $param.value);}
  | 
  'reduce' param{$value = Buffer.reduce($buff, $param.value);}
  | 
  'post' {$value = Buffer.post($buff);}
  | 
  'play' (p=param {$value = Buffer.play($buff,$p.value);}| {$value = Buffer.play($buff,1.0);})
  | 
  'loop' a=param {$value = Buffer.loop($buff, $a.value);}
  | 
  'fill' a=param {$value = Buffer.fill($buff, $a.value);}
  | 
  'tune' a=param {$value = Buffer.tune($buff, $a.value);}
  ;

/* Los parametros pueden ser numeros, o sino, sin parentesis*/
param returns [Double value]:	PR_START n=num PR_END {$value = getDouble($n.text);}
	;

/* Los numeros aceptados por la gramatica */
num returns  [String text]:
	(ADD{$text = $ADD.text;} 
    | SUB {$text = $SUB.text;} 
    | {$text = "";}) i1=INT {$text = $text + $i1.text;} 
        (PERIOD i2=INT {$text = $text + $PERIOD.text + $i2.text;})?
	;




\end{lstlisting}
\newpage
\subsection{Buffer.java}
\begin{lstlisting}[language=Java]
import java.util.ArrayList;
import java.lang.Math;
import javax.sound.sampled.AudioFormat;
import javax.sound.sampled.AudioSystem;
import javax.sound.sampled.SourceDataLine;


public class Buffer {
    public static int sampling_rate = 44100;
    public static int beat = sampling_rate / 12;

    public static ArrayList<Double> resize
            (ArrayList<Double> a, double number) {
        ArrayList<Double> buff = new ArrayList<Double>();
        for (int i = 0; i < number; i++) {
            buff.add(a.get(i % a.size()));
        }
        return buff;
    }

    public static ArrayList<Double> buffer(double number) {
        ArrayList<Double> buff = new ArrayList<Double>();
        for (int i = 0; i < beat; i++) {
            buff.add(number);
        }
        return buff;
    }

    public static ArrayList<Double> sin (double c, double a) {
        ArrayList<Double> buff = new ArrayList<Double>();
        double x = (c * 2 * Math.PI)/beat;

        for (int i = 0; i < beat; i++) {
            buff.add(a * Math.sin(i * x));
        }
        return buff;
    }

    public static ArrayList<Double> sil() {
        ArrayList<Double> buff = new ArrayList<Double>();
        double cero = 0;
        for (int i = 0; i < beat; i++) {
            buff.add(cero);
        }
        return buff;
    }

    public static ArrayList<Double> lin (double a, double b) {
        ArrayList<Double> buff = new ArrayList<Double>();
        double diff = (b - a)/(beat - 1);
        for (int i = 0; i < beat; i++) {
            buff.add(a + diff * i);
        }
        return buff;
    }

    public static ArrayList<Double> noi (double a) {
        ArrayList<Double> buff = new ArrayList<Double>();
        for (int i = 0; i < beat*a; i++) {
            buff.add(Math.random()*2 - 1);
        }
        return buff;
    }

    public static void music_play
            (ArrayList<Double> notes, Double speed) {
        byte[] buf = new byte[1];
        double sampling_freq =sampling_rate;

        try {
            AudioFormat af = 
                new AudioFormat((float) sampling_freq, 8, 1, true, false);
            SourceDataLine sdl = AudioSystem.getSourceDataLine(af);
            sdl = AudioSystem.getSourceDataLine(af);
            sdl.open(af);
            sdl.start();
            System.out.format("Sound duration: %.3f s%n", 
                    (1/12.0)* (((double)notes.size())/beat)/speed);
            assert(speed>0.0);
            double true_notes_length = notes.size()/speed;

            for(int j = 0 ; j < true_notes_length; j++) {
                Double note;
                int index_base = (int) ((double)j*speed);
                int index_top = (int) (((double)(j+1))*speed);
                if(index_base>=notes.size()) index_base=notes.size()-1;
                if(index_top>=notes.size()) index_top=notes.size()-1;
                Double note_left = notes.get(index_base)*100.0;
                Double note_right = notes.get(index_top)*100.0;
                Double coef = (index_base-index_top)/(note_left-note_right) ;
                note = coef * note_left + (1.0-coef) * note_right;
                buf[0]= note.byteValue();
                sdl.write(buf, 0, 1);

            }
            sdl.drain();
            sdl.stop();
            sdl.close();
        } catch (Exception ex) {
            System.out.println
                ("Error en la reproduccion: " + ex.getCause() + 
                " \n Detalles: "  + ex.getMessage());
        }
    }

    public static ArrayList<Double> resample(
            ArrayList<Double> buff, int L) {
        ArrayList<Double> r = new ArrayList<Double>();
        int size = buff.size();
        for (int i = 0; i < L; i++) {
            r.add(buff.get(i * size / L));
        }
        return r;
    }

    public static ArrayList<Double> tune
            (ArrayList<Double> buff, int P) {
        int size = buff.size();
        return resample(buff,Utils.doubleToInt(size * Math.pow(2.0, -P/12.0)));
    }

    public static ArrayList<Double> resize
            (ArrayList<Double> buff, int L) {
        ArrayList<Double> r = new ArrayList<Double>();
        int size = buff.size();
        for (int i = 0; i < L; i++) {
            r.add(buff.get(i % size));
        }
        return r;
    }

    public static ArrayList<Double> reduce
            (ArrayList<Double> buff, int N) {
        int L = beat * N;
        if(buff.size() > L) {
            return resample(buff, L);
        } else {
            return buff;
        }
    }

    public static ArrayList<Double> expand(ArrayList<Double> buff, int N) {
        int L = beat * N;
        if(buff.size() < L) {
            return resample(buff, L);
        } else {
            return buff;
        }
    }

    public static ArrayList<Double> fill(ArrayList<Double> buff, int N) {
        ArrayList<Double> r = new ArrayList<Double>();
        int L = beat * N;
        int size = buff.size();
        for (int i = 0; i < L; i++) {
            if(i < size) {
                r.add(buff.get(i));
            } else {
                r.add(0.0);
            }
        }

        return r;
    }

    public static ArrayList<Double> oper
            (String op, ArrayList<Double> buffA, ArrayList<Double> buffB) {
        ArrayList<Double> a = buffA.size() < buffB.size() ? 
                                resize(buffA, buffB.size()) : 
                                buffA;
        ArrayList<Double> b = buffB.size() < buffA.size() ? 
                                resize(buffB, buffA.size()) : 
                                buffB;

        if (op.equals("+")) return sum(a,b);
        if (op.equals("-")) return sub(a,b);
        if (op.equals("*")) return mul(a,b);
        if (op.equals("/")) return div(a,b);
        if (op.equals("m")) return mix(a,b);
        return null;
    }

    public static ArrayList<Double> sum
            (ArrayList<Double> a, ArrayList<Double> b) {
        ArrayList<Double> r = new ArrayList<Double>();
        for (int i = 0; i < a.size(); i++) r.add(a.get(i) + b.get(i));
        return r;
    }

    public static ArrayList<Double> sub
            (ArrayList<Double> a, ArrayList<Double> b) {
        ArrayList<Double> r = new ArrayList<Double>();
        for (int i = 0; i < a.size(); i++) r.add(a.get(i) - b.get(i));
        return r;
    }

    public static ArrayList<Double> mul
            (ArrayList<Double> a, ArrayList<Double> b) {
        ArrayList<Double> r = new ArrayList<Double>();
        for (int i = 0; i < a.size(); i++) r.add(a.get(i) * b.get(i));
        return r;
    }

    public static ArrayList<Double> div
            (ArrayList<Double> a, ArrayList<Double> b) {
        ArrayList<Double> r = new ArrayList<Double>();
        for (int i = 0; i < a.size(); i++) r.add(a.get(i) / b.get(i));
        return r;
    }

    public static ArrayList<Double> mix
            (ArrayList<Double> a, ArrayList<Double> b) {
        ArrayList<Double> r = new ArrayList<Double>();
        for (int i = 0; i < a.size(); i++) r.add((a.get(i) + b.get(i))/2);
        return r;
    }

    public static ArrayList<Double> expand
            (ArrayList<Double> buff, Double n) {
        Double total = beat*n;
        if (buff.size() < total) {
            return resize(buff, total);
        } else {
            return buff;
        }
    }

    public static ArrayList<Double> reduce
            (ArrayList<Double> buff, Double n) {
        Double total = beat*n;
        if (buff.size() > total) {
            return resize(buff, total);
        } else {
            return buff;
        }
    }

    public static ArrayList<Double> post
            (ArrayList<Double> buff) {
        for(int i = 0; i<buff.size(); i += beat) {
            System.out.print(buff.get(i + beat/2) + " ");
        }
        System.out.println("");
        return buff;
    }

    public static ArrayList<Double> play
            (ArrayList<Double> buff, Double speed) {
        music_play(buff, speed);
        return buff;
    }

    public static ArrayList<Double> loop 
            (ArrayList<Double> buff, double times) {
        ArrayList<Double> localbuff = new ArrayList<Double>();
        int fulltimes=(int)times;
        for(int i = 0; i< fulltimes; i++)
            localbuff.addAll(buff);
        int partial = (int) ((times - fulltimes)*(double)buff.size());
        for(int i = 0; i< partial; i++)
            localbuff.add(buff.get(i));

        return localbuff;
    }

    public static ArrayList<Double> fill
            (ArrayList<Double> buff, Double size) {
        ArrayList<Double> localBuff = new ArrayList<Double>();

        Double total = size * beat;
        Double zeros = total > buff.size() ? total - buff.size() : 0 ;

        for(int i = 0; i < buff.size(); i++) localBuff.add(buff.get(i));

        for(int i = 0; i < zeros; i++) localBuff.add(0.0);

        return buff;
    }

    public static ArrayList<Double> tune
            (ArrayList<Double> buff, Double pitch) {
        return resample(buff, 
                    (int)(buff.size()*(Math.pow(Math.pow(2, 1/12.0),-pitch))));
    }

}



\end{lstlisting}

\end{document}
