Decisiones tomadas
==================

Decidimos tomar las siguientes consideraciones:
- Todas las funciones  pueden tomar 0, 1 o 2 parametros, tomando valores por defecto en caso de que
    sean menos de 2 parametros. En el caso de tener 0 parametros, no se escriben los parentesis '()'
- El whitespace de los programas es eliminado y no considerado, con excepcion del caso de las // que
    eliminan lo siguiente de esa linea.
- La gramatica definida es LL(1) dado que para dos producciones con misma parte izquierda, no hay 
    son disjuntos los simbolos directrices generados.


Conclusion
===========

A traves de este trabajo, pudimos apreciar las diferencias entre lexers y parsers. Aunque tomamos un
subconjunto de las reglas necesarias para escribir una gramatica que sea entendible por un compilador,
hace falta un poco mas de trabajo para que sea entendible por un compilador como \texttt{ANTLR} o \texttt{yacc}. 
Quizas haber hecho que sea LL(1) lo hace un poco mas ilegible que si no nos hubieramos puesto esa restriccion.



